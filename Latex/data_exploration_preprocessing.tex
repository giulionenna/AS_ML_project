We first import the dataset and encode each categorical feature using \textit{one-hot encoding}. This means that numerical features will be left unchanged while for each categorical feature the process is the following:
\begin{description}
    \item[1.] Determine all the distinct values of that feature (categories)
    \item[2.] For each category generate a new binary column
    \item[3.] Assign values to the binary columns according to categories featred in each line.   
\end{description}
\begin{center}
    \begin{tabular}{|c|}
        \hline
        feature \\
        \hline
        \hline
        category 1 \\
        \hline
        category 2 \\
        \hline
        category 3 \\
        \hline
        category 2 \\
        \hline
    \end{tabular}
    \quad
    \begin{tabular}{|c|c|c|}
        \hline
        category 1 & category 2 & category 3\\
        \hline
        \hline
        1 & 0 & 0 \\
        \hline
        0 & 1 & 0 \\
        \hline 
        0 & 0 & 1 \\
        \hline
        0 & 1 & 0 \\
        \hline
    \end{tabular}
        
\end{center}
